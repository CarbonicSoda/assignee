\documentclass[12pt]{report}
\usepackage{times}
\usepackage{xcolor}
\usepackage{hyperref}
\hypersetup{colorlinks=true, linkcolor=teal}
\usepackage{graphicx}
\usepackage{indentfirst}
\usepackage{dirtree}
\setlength{\DTbaselineskip}{20pt}

\title{SBA Web Application Report}
\author{David Y.C. Wu}
\date{Last Revised \today}

\begin{document}

\maketitle

\textbf{Introduction}
\vspace{1 em}

This report focuses on Assignee, a web application born in response to the SBA task:
Design a web application that implements an assignment system.
\vspace{1 em}

This report contains the following chapters of different aspects:
\begin{itemize}
	\item \hyperref[overview]{Overview}:\\
	      Project objectives and structure;
	\item \hyperref[data-layer]{Data Layer}:\\
	      Relational database design and implementation;
	\item \hyperref[application-layer]{Application Layer}:\\
	      Site back-end components design and implementation;
	\item \hyperref[presentation-layer]{Presentation Layer}:\\
	      Site front-end components design and implementation;
	\item \hyperref[security]{Security}:\\
	      User authentication and cross-layer interaction security;
	\item \hyperref[accessibility]{Accessibility}:\\
	      Site accessibility considerations and localization (l10n);
	\item \hyperref[performance]{Performance}:\\
	      Application performance optimizations;
	\item \hyperref[quality-assurance]{Quality Assurance}:\\
	      Application unit tests and integration tests;
	\item \hyperref[tools-used]{Tools Used}:\\
	      Applications, extensions, packages and libraries used during development;
\end{itemize}

\tableofcontents
\newpage

\chapter{Overview} \label{overview}

\section{Project Objectives} \label{overview.project-objectives}

The next few pages is a brief overview of what could be done by users of different roles.
A few systems are decoupled to handle complicated application logic.
As of now we have the 3 core systems: the User System, the Team System and the Assignment System.
\vspace{1 em}

Only workflow related points are listed, GUI related points are not.

Only roughly describes each task (e.g. login),
the rest of the details (e.g. via username? with auth? etc.) are given later in the report.
\vspace{1 em}

Symbolic notes:

...... $\rightarrow$ \{A\} means the user who does the action now attains role A.

\{B\} $\rightarrow$ \{C\} means the action will make a user with role B to attain role C.

\{D\} $\in$ \{E\} means the role D inherits from role E,
users with role D will be able to do anything role E is able to do.

\newpage
\subsection{User System} \label{overview.project-objectives.user-system}

Roles: \{Site Visitors\}, \{Logged-in Users\}

\begin{itemize}
	\item \{Site Visitors\}
	      \begin{itemize}
		      \item Register for accounts;\null\hfill $\rightarrow$ \{Logged-in Users\}
		      \item Log-in to existing accounts;\null\hfill $\rightarrow$ \{Logged-in Users\}
	      \end{itemize}
	\item \{Logged-in Users\}
	      \begin{itemize}
		      \item Alter password;
		      \item Alter username or email;
		      \item Alter other settings;
		      \item Log-out of account;\null\hfill $\rightarrow$ \{Site Visitors\}
		      \item Delete the account;\null\hfill $\rightarrow$ \{Site Visitors\}
	      \end{itemize}
\end{itemize}

The User System is simply a casual user account system you will find in any web application.
Visitors are allowed to create accounts and ones with an account could log-in and manipulate the account.
\vspace{1 em}

Authentication techniques are implemented to assist this system, which details are given in later chapters.
\vspace{1 em}

The User System is the core of all systems possibly involved in this application,
including the \hyperref[overview.project-objectives.team-system]{Team System}
and the \hyperref[overview.project-objectives.assignment-system]{Assignment System},
and any systems that might be added in the future (e.g. Posts and Comments System, WebRTC Instant Messaging System).

\newpage
\subsection{Team System} \label{overview.project-objectives.team-system}

Roles: \{Users\}, \{Team Members\}, \{Team Monitors\}, \{Team Owners\}

\begin{itemize}
	\item \{Users\}
	      \begin{itemize}
		      \item Join teams;\null\hfill $\rightarrow$ \{Team Members\}
		      \item Create teams;\null\hfill $\rightarrow$ \{Team Owners\}
	      \end{itemize}
	\item \{Team Members\} $\in$ \{Users\}
	      \begin{itemize}
		      \item Alter per-team settings;
		      \item Leave the team;\null\hfill $\rightarrow$ \{Users\}
		      \item Comment on messages*;
	      \end{itemize}
	\item \{Team Monitors\} $\in$ \{Team Members\}
	      \begin{itemize}
		      \item Invite team members;\null\hfill \{Users\} $\rightarrow$ \{Team Members\}
		      \item Kick team members;\null\hfill \{Team Members\} $\rightarrow$ \{Users\}
		      \item Post messages*;
	      \end{itemize}
	\item \{Team Owners\} $\in$ \{Team Monitors\}
	      \begin{itemize}
		      \item Alter team name;
		      \item Alter team description;
		      \item Alter other team settings;
		      \item Appoint team monitors;\null\hfill \{Users\} $\rightarrow$ \{Team Monitors\}
		      \item Remove team monitors;\null\hfill \{Team Monitors\} $\rightarrow$ \{Users\}
		      \item Delete the team;\null\hfill $\rightarrow$ \{Users\}
	      \end{itemize}
\end{itemize}

Slightly isomorphic to the well-known Google Classroom,
users are allowed to create or join teams.
Team owners could (de-)appoint team monitors to help manage the team.
Roles of different hierarchies could manipulate the team to different degree and extent.
\vspace{1 em}

The *post and comment features are not implemented, at least in the initial version of the web app,
as it is deemed to be of lower priority. It will be decoupled from this system if implemented.
\vspace{1 em}

The Team System is the core of the \hyperref[overview.project-objectives.assignment-system]{Assignment System}.

\newpage
\subsection{Assignment System} \label{overview.project-objectives.assignment-system}

Roles: \{Team Owners\}, (\{Team Members\},) \{Assignees\}

\begin{itemize}
	\item \{Team Owners\}
	      \begin{itemize}
		      \item Author assignments;
		      \item Add attachments to assignments;
		      \item Assign assignments;\null\hfill \{Team Members\} $\rightarrow$ \{Assignees\}
		      \item De-assign assignments;\null\hfill \{Assignees\} $\rightarrow$ \{Team Members\}
		      \item Grade, comment and return submissions;
	      \end{itemize}
	\item \{Assignees\} $\in$ \{Team Members\}
	      \begin{itemize}
		      \item Submit assignments;
		      \item Add attachments to submissions;
	      \end{itemize}
\end{itemize}

The Assignment System is the core feature we are asked to complete.
Works intuitively, team owners are allowed to assign assignments to some team members, with a deadline.
The assignees are allowed to hand in their submission,
after which the team owner could grade it, comment on it and return it to assignees.
It is that straightforward.
\vspace{1 em}

An attachment system is designed to handle attachment files, to allow Team Owners/Assignees
to attach more files to their assignments/submissions along the way.
More details will be given in later chapters.

\newpage
\section{Project Structure} \label{overview.project-structure}

\subsection{Folder Structure} \label{overview.project-structure.folder-structure}

To keep a clear separation of concern, the web app is separated into
the Data Layer, the Application Layer and the Presentation Layer.
This could be seen in the project folder structure.
The folder structure is subject of frequent change, but the overall shape will not differ vastly across revisions.
\vspace{1 em}

\dirtree{%
	.1 /.
	.2 design/.
	.3 database/.
	.4 assigneedb.sql.
	.4 ....
	.3 report.tex.
	.3 ....
	.2 site/.
	.3 backend/.
	.4 ....
	.3 frontend/.
	.4 ....
}
\vspace{1 em}

The directory database/ is kept under design/ since the hosted database itself is independent of the project folder,
so only the distribution .sql file that would be used to instantiate the database is kept.

\subsection{Repository} \label{overview.project-structure.repository}

The whole project folder has a repository and available at
\href{https://github.com/CarbonicSoda/assignee}{GitHub/Carbonic\-Soda/assignee}.
(If you cannot open the repo, I might have set visibility to private.)

\chapter{Data Layer} \label{data-layer}


\chapter{Application Layer} \label{application-layer}


\chapter{Presentation Layer} \label{presentation-layer}


\chapter{Security} \label{security}


\chapter{Accessibility} \label{accessibility}


\chapter{Performance} \label{performance}


\chapter{Quality Assurance} \label{quality-assurance}


\chapter{Tools Used} \label{tools-used}


\end{document}
