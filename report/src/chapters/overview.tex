\chapter{Overview}
\label{overview}

\section{Introduction}
\label{overview.intro}

This report concerns Assignee, a full stack web application for the ICT SBA task
of implementing an assignment system.

In this chapter, we would cover the capabilities of Assignee users, and the core
systems of design that would guide later sections. We would also touch side systems
built upon the core. At the end of the chapter, resources for reference and validation
would be provided.

\section{Capabilities}
\label{overview.capab}

This section aims to describe the roles of different users of Assignee and what
they can accomplish via the app. The section would be a high level overview of user
workflow, thus omitting other capabilities such as those related to
accessibility, which would be covered in later chapters.

Symbolic notes:
\begin{itemize}
	\item Role inheritance is denoted by $Child\in{}Parent$

	\item Role transitions are denoted by $Old\xrightarrow{Transition}New$

	\item Role transitions of $\geq{}1$ users are post-fixed with $*$
\end{itemize}

Role inheritance means all actions that could be done by the role $Parent$ could
be done by the role $Child$. A role transition happens when the user with the
role $Old$ performs a certain action and thus attains the $New$ role.

Items marked with a dagger \textdagger{} are not implemented in the actual application,
at least in initial versions. This is primarily because they have lower priority,
but details would be given as needed in paragraphs starting with daggers.

\subsection{User System}
\label{overview.capab.user}

Roles: $(Visitor, User)$

$Visitor$
\begin{itemize}
	\item $Visitor\xrightarrow{Signup}User$

	\item $Visitor\xrightarrow{Signin}User$
\end{itemize}

$User$
\begin{itemize}
	\item Change sign-in email

	\item Change sign-in password

	\item Change display name

	\item Change other settings \textdagger{}

	\item $User\xrightarrow{Signout}Visitor$

	\item $User\xrightarrow{Revoke}Visitor$
\end{itemize}

The very basics of the application is the user system which manages user accounts.

An account can be identified solely by an email. The majority nowadays own an
email (students/employers own school-/corp-domain emails). Therefore, visitors will
not have to think of unique names to sign-up but to use their email right away.
After which they can choose display names that resonate with them.

\textdagger{} Other settings are not implemented in initial versions. For one, Assignee
is rather opinionated, with everything tuned for the good. For two, the
complexities of Assignee in its current phase is low, writing one would over-complicate
the app. However, the database table for it would still be given in later chapters
for reference of potential implementation.

%MO TODO move to other chapters/sections, doesnt fit here
% \textdagger{} Albeit using password solely can harm security, email 2FA is abandoned
% so that reviewers can run the app without access to email API keys, which would
% otherwise pose security threats. The database table for it would still be included
% in later chapters for reference, having simple implementations available for
% actual deployments.

\subsection{Team System}
\label{overview.capab.team}

Roles: $(User, Member, Admin\dagger{}, Owner)$

$User$
\begin{itemize}
	\item $User\xrightarrow{Join}Member$

	\item $User\xrightarrow{Create}Owner$
\end{itemize}

$Member\in{}User$
\begin{itemize}
	\item $Member\xrightarrow{Leave}User$
\end{itemize}

$Admin\in{}Member\dagger{}$
\begin{itemize}
	\item Invite members ($User\xrightarrow{Invite*}Member$)

	\item Kick members ($Member\xrightarrow{Kick*}User$)
\end{itemize}

$Owner\in{}Admin$
\begin{itemize}
	\item Change team name

	\item Change team description

	\item Appoint admin ($Member\xrightarrow{Appoint*}Admin$) \textdagger{}

	\item Dismiss admin ($Admin\xrightarrow{Dismiss*}Member$) \textdagger{}

	\item $Owner\xrightarrow{Disband}User$ ($Member\xrightarrow{Dismiss*}User$)
\end{itemize}

Built upon the user system is the team system which manages user-created teams.

Assignee users are free to create or join teams as they need, and team owners
are allowed to appoint team admins to help with management. To some extent, the
logic is similar to Google Classroom.

\textdagger{} The $Admin$ role is not directly implemented in initial versions to
speed up project iteration. However, the $Owner$ role, which is inherited from
$Admin$, would still have the privileges of it. The database table for the role
would still be present in later chapters for reference of potential
implementation.

\subsection{Assignment System}
\label{overview.capab.assign}