\chapter{Overview}
\label{overview}

\section{Introduction}
\label{overview.intro}

This report concerns Assignee, a full stack web application for the ICT SBA task
of implementing an assignment system.

Throughout the report, items marked with a dagger \textdagger{} are not
implemented in the actual application, at least in initial versions. This is
primarily because they have lower priority, but details would be given as needed
in paragraphs starting with daggers.

In this chapter, we would cover the capabilities of Assignee users, and the core
systems of design that would guide later sections. We would also touch side
systems built upon the core. At the end of the chapter, resources for reference
and validation would be provided.

\section{Capabilities}
\label{overview.capab}

This section aims to describe the roles of different users of Assignee and what they
can accomplish via the app. The section would be a high level overview of user
workflow, thus omitting other capabilities such as those related to accessibility,
which would be covered in later chapters.

Symbolic notes:
\begin{itemize}
	\item Role inheritance is denoted by $Child\in{}Parent$

	\item Role transitions are denoted by $Old\xrightarrow{Transition}New$

	\item Role transitions of $\geq{}1$ users are post-fixed with $*$
\end{itemize}

Role inheritance means all actions that could be done by the role $Parent$ could
be done by the role $Child$. A role transition happens when the user with the role
$Old$ performs a certain action and thus attains the $New$ role.

\subsection{User System}
\label{overview.capab.user}

Roles: $(Visitor, User)$

$Visitor$
\begin{itemize}
	\item $Visitor\xrightarrow{Signup}User$

	\item $Visitor\xrightarrow{Signin}User$
\end{itemize}

$User$
\begin{itemize}
	\item Change sign-in email

	\item Change sign-in password

	\item Change display name

	\item Change other settings \textdagger{}

	\item $User\xrightarrow{Signout}Visitor$

	\item $User\xrightarrow{Revoke}Visitor$
\end{itemize}

The very basics of the application is the user system which manages user
accounts.

An account can be identified solely by an email. The majority nowadays own an email
(students/employers own school-/corp-domain emails). Therefore, visitors will
not have to think of unique names to sign-up but to use their email right away. After
which they can choose display names that resonate with them.

\textdagger{} Other settings are not implemented in initial versions. For one,
Assignee is rather opinionated, with everything tuned for the good. For two, the
complexities of Assignee in its current phase is low, writing one would over-complicate
the app. However, the database table for it would still be given in later
chapters for reference of potential implementation.

%MO TODO move to other chapters/sections, doesnt fit here
% \textdagger{} Albeit using password solely can harm security, email 2FA is abandoned
% so that reviewers can run the app without access to email API keys, which would
% otherwise pose security threats. The database table for it would still be included
% in later chapters for reference, having simple implementations available for
% actual deployments.

\subsection{Team System}
\label{overview.capab.team}

Roles: $(User, Member, Admin\dagger{}, Owner)$

$User$
\begin{itemize}
	\item $User\xrightarrow{Join}Member$

	\item $User\xrightarrow{Create}Owner$
\end{itemize}

$Member\in{}User$
\begin{itemize}
	\item $Member\xrightarrow{Leave}User$
\end{itemize}

$Admin\in{}Member\dagger{}$
\begin{itemize}
	\item Invite members ($User\xrightarrow{Invite*}Member$)

	\item Kick members ($Member\xrightarrow{Kick*}User$)
\end{itemize}

$Owner\in{}Admin$
\begin{itemize}
	\item Change team name

	\item Change team description

	\item Appoint admins ($Member\xrightarrow{Appoint*}Admin$) \textdagger{}

	\item Dismiss admins ($Admin\xrightarrow{Dismiss*}Member$) \textdagger{}

	\item $Owner\xrightarrow{Disband}User$ ($Member\xrightarrow{Dismiss*}User$)
\end{itemize}

Built upon the user system is the team system which manages user-created teams.

Assignee users are free to create or join teams as they need, and team owners are
allowed to appoint team admins to help with management. To some extent, the logic
is similar to Google Classroom.

\textdagger{} The $Admin$ role is not directly implemented in initial versions
to speed up project iteration. However, the $Owner$ role, which is inherited from
$Admin$, would still have the privileges of it. The database table for the role would
still be present in later chapters for reference of potential implementation.

\subsection{Assignment System}
\label{overview.capab.assign}

Roles: $(Owner, (Member), Assignee)$

$Owner$
\begin{itemize}
	\item Create assignments

	\item Add assignment attachments

	\item Assign assignment ($Member\xrightarrow{Assign*}Assignee$)

	\item Revoke assignment ($Assignee\xrightarrow{Revoke*}Member$)

	\item Return submissions with feedback
\end{itemize}

$Assignee\in{}Member$
\begin{itemize}
	\item Add submission attachments

	\item Submit submission
\end{itemize}

The system that fulfills the core objective is the assignment system which keeps
track of assignments and submissions.

Team owners are allowed to create assignments with descriptions, assign them to a
selection of team members with a deadline, and later return grades or comments. Attachments
would be preserved across different sessions, without the need to re-attach all
previously attached files.

\subsection{Side Systems \textdagger{}}
\label{overview.capab.side}

Other systems would all be built upon the previous three systems. This would
become potential extensions of Assignee, like team messaging channels with WebSockets
or user instant messaging system with WebRTC.

\textdagger{} Side systems are after all relatively minor as its name suggests. To
speed up project completion and iterations, side systems are not featured in
initial versions. In fact, not even database tables were designed for them. However,
potential implementations would not be sophisticated given the foundation of the
core systems.

\section{Resources}
\label{overview.src}

Several resources are listed here for reference and validation of the SBA invigilators.

\subsection{Repository}
\label{overview.src.repo}

The entire project is hosted in a repository, available at \href{https://github.com/CarbonicSoda/assignee}{Repository}
for inspection.

Modular approach is taken for clarity and separation of concern. The report/ directory
contains the \LaTeX{} source of this report; The database/ directory contains
the SQL dump file that is used to instantiate the backing database; The site/
directory contains the actual web application, further separated into backend/
of application logic and frontend/ of presentation logic.

\subsection{Review}
\label{overview.src.review}

Invigilators may not have dependencies of Assignee installed, so prebuilt archives
are provided in \href{https://github.com/CarbonicSoda/assignee/releases}{Releases}.
To run the app, unzip the archive to somewhere convenient and execute the binary
(app.exe for Windows, for other platforms pick corresponding archives). Database
records would be preserved in the app.db file.